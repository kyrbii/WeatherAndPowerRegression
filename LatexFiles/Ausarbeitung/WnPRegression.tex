%File: formatting-instructions-latex-2023.tex
\nocite{*}
%release 2023.0
\documentclass[letterpaper]{article} % DO NOT CHANGE THIS
\usepackage{aaai23}  % DO NOT CHANGE THIS
\usepackage{times}  % DO NOT CHANGE THIS
\usepackage{helvet}  % DO NOT CHANGE THIS
\usepackage{courier}  % DO NOT CHANGE THIS
\usepackage[hyphens]{url}  % DO NOT CHANGE THIS
\usepackage{graphicx} % DO NOT CHANGE THIS
\urlstyle{rm} % DO NOT CHANGE THIS
\def\UrlFont{\rm}  % DO NOT CHANGE THIS
\usepackage{natbib}  % DO NOT CHANGE THIS AND DO NOT ADD ANY OPTIONS TO IT
\usepackage{caption} % DO NOT CHANGE THIS AND DO NOT ADD ANY OPTIONS TO IT
\frenchspacing  % DO NOT CHANGE THIS
\setlength{\pdfpagewidth}{8.5in}  % DO NOT CHANGE THIS
\setlength{\pdfpageheight}{11in}  % DO NOT CHANGE THIS
%
% These are recommended to typeset algorithms but not required. See the subsubsection on algorithms. Remove them if you don't have algorithms in your paper.
\usepackage{algorithm}
\usepackage{algorithmic}
\usepackage{bibentry}


\pdfinfo{
/TemplateVersion (2023.1)
}

\setcounter{secnumdepth}{0} %May be changed to 1 or 2 if section numbers are desired.

\title{Stromerzeugung bei unterschiedlichen Wetterlagen\\-- Datenaufbereitung und Analyse zur Korrelation Mithilfe von Regressionsalgorithmen in Weka}
\author {
    % Authors
    Korbinian Eller,
    Kay Gietenbruch
}

\begin{document}
\maketitle
\begin{abstract}
    In dieser Ausarbeitung wird das Vorgehen beim Sammeln von Daten, deren Aufbereitung und die Analyse mithilfe der Regressionsalgorithmen in Weka erläutert und die Ergebnisse dokumentiert.
    
    Ziel ist es, durch Wetterdaten des Deutschen Wetterdienstes (DWD) auf Stromerzeugungszahlen der erneuerbaren Energien Solar und Wind (größte Beeinflussung durch Wetter) zu schließen.
\end{abstract}
\section*{Idee}
    Zur gegebenen Themenstellung "Einen Themenbereich der KI vertiefen" war eine erste Idee die Klassifikation von Datensätzen. Dies war sehr ähnlich einer der letzten Aufgabenstellungen der Übungsstunden des Fachs. Es war vor allem aufgrund der Implementierung in der Programmiersprache C und damit dem greifbar machen des Algorithmus interessant.

    Der Gedanke der Klassifikation war schnell gefestigt, nun fehlten noch die Daten, mit denen gearbeitet werden soll. Über Daten wie Covid-19 Erkrankungszahlen, Bußgeldbescheide, Denkmalstandorte oder Amazon Personendaten ist eine Idee herausgestochen.

    "Es wäre doch interessant, Wetterdaten und Stromdaten in Korrelation miteinander zu bringen und so die Stromerzeugung anhand des vorherrschenden Wetters deuten zu können." Und das war dann das Noema, mit dem fortgefahren werden sollte. Das Ziel ist, eine repräsentative Menge der Daten zu sammeln, um eine Klassifikation sinnvoll ausführen zu können. Trotz all dem war die Beschränkung der Daten auch ausschlaggebend. Festgelegt wurde sich dann auf den Zeitraum eines Jahres und wegen der Zeit, in der die Daten gesammelt wurden (Ende 2023) war das Jahr 2022 passend für die Aufgabe.
    
    Das Projekt konnte indessen in mehrere Schritte unterteilt werden: die Daten sammeln, die Daten aufbereiten, den Klassifikator programmieren, testen und verbessern und die Arbeit dokumentieren.

    Nach Rücksprache mit dem Betreuer des Projekts Herrn Prof. Dr. Baumann wurde jedoch klar, dass eine Klassifizierung der Daten nicht das geeignetste Modell für die Analyse der Korrelation ist. So wurde der Plan neu geschrieben und eine Analyse mithilfe von Regressionsalgorithmen in dem Machine Learning Programm Weka stand ab dem Zeitpunkt im Vordergrund. Dafür sollten die Daten noch angepasst und dann mit Weka und den Regressionsalgorithmen experimentiert werden. In dem Sinne, dass das am bestgeeignete Modell zur Regression gefunden wird!
\section*{Datenrecherche}
        Die Daten, mit denen die Regression in Weka betrieben werden soll, müssen zuerst zusammengetragen werden. Sowohl die Daten des DWDs, respektive die Wetterdaten, als auch die der Bundesnetzagentur(BNetzA), welche die Daten der Stromerzeugung bereitstellt, stehen leider nicht auf der jeweiligen Websites als Download in Form von z.B. ".csv" oder ".xls" Dateiformaten zur Verfügung. Hier war also eine andere Herangehensweise gefragt. Nun muss zwischen den Daten des DWD und denen der BNetzA unterschieden werden, da diese in unterschiedlicher Form vorliegen und somit auch einem unterschiedlichen Sammel-Prozess unterlaufen sind.

        Es ist anzumerken, dass dieser Arbeit einige Dateien beigelegt sind, unter denen sich auch ".md" Dateien befinden, in welchen sehr speziell auf die einzelnen Schritte und die Struktur eingegangen wurde.


        \subsection*{BNetzA Daten}
        Die Bundesnetzagentur stellt ihre recht simplen Daten (Im Vergleich zum DWD) auf der Website "Strommarktdaten" sehr anschaulich in einem Flächendiagramm dar. Aus solch einem Diagramm Werte in eine Datenbank zu übertragen wäre aber bei den stündlichen Werten von dem ganzen Jahr 2022, sprich 8760 Zeilen einer CSV Datei und jeweils zwölf Datentypen undenkbar. Glücklicherweise liefert die BNetzA nicht eine Grafik an die Website, sondern die stündlichen Daten in mehreren Dateien vom Typ ".json" an den Browser, wo dann ein Diagramm erstellt wird.

        Das heißt, durch einen einfachen "cURL" Befehl auf der Kommandozeile, kann eine solche Datei, wenn der vollständige Name bekannt ist, heruntergeladen werden. Die Namen bestehen aus einer Zahlenkodierung der unterschiedlichen Daten-Arten (z.B. Kernenergie - 1224, Wind Onshore - 4067), dem Kürzel DE für Deutschland, der "Auflösung" der Daten (z.B. hour, day) und dem Epoch Zeitstempel für den frühesten aufgezeichneten Wert in dieser Datei.
        Diese Dateien erfassen immer eine ganze Woche, das heißt die Epoch Zeit Codes unterscheiden sich immer um genau 604.800.000 Millisekunden, da dies genau 7 Tage abbildet. So konnte das Jahr 2022 schnell in Epoch Codes abgebildet werden und es stand fest welche Dateien benötigt werden, da alle Type erzeugten Stroms geladen wurden, weil es relativ problemlos machbar war.
        
        Der Aufbau der angesprochenen ".json" Dateien ist ein Array mit 180 Datenpunkten, die jeweils zwei Werte haben. Einmal den Epoch Zeit Code und die Höhe des gemessenen Stroms des jeweiligen Typs in MWh. Am Anfang der Datei beschreibt eine Versionsnummer und ein Zeitpunkt (ebenfalls Epoch) die Erstellung der Datei.
        Der nächste Schritt war nun, die Dateien in einem iterierendem Python Programm auf der eigenen Festplatte zu speichern. Einziges Problem hierbei war nur, dass die Dateien immer montags um 0 Uhr beginnen, aber das Jahr 2022 nicht auf einen Montag begonnen bzw. auf einen Sonntag geendet ist. Folglich sind mehr Dateien, als es eigentlich braucht, um ein Jahr abzubilden, geladen worden.
 
    \subsection*{DWD - Daten}
        Der Deutsche Wetterdienst stellt seine Daten auf einem opendata Server zu Verfügung. Nachdem sich in der Dateistruktur zurechtgefunden wurde, sind auch die ersten Probleme aufgekommen:
        \begin{itemize}
            \item Welche Wetterdaten sollen benutzt werden?
            \item Wie werden die gewünschten Zeiträume gefunden?
            \item Welche Dateien enthalten welche Daten?
        \end{itemize}
        Einiges, das im ersten Moment sehr unklar erschien! Die Lösungen waren dann wie folgt:
            \subsubsection*{Datenart}
                Es wurde sich auf Eigenschaften beschränkt, die aus subjektiver und ungeschulter Sicht stark in die Stromproduktion als Faktoren miteinfließen. In erster Linie handelt es sich um erneuerbare Energien, also galt es bestimmte Naturphänomene, die Photovoltaik-Anlagen oder Windkraftanlagen beeinflussen, auszumachen. Dies sind:
                \begin{itemize}
                    \item Lufttemperatur, Feuchtigkeit
                    \item Bedeckungsgrad des Himmels
                    \item Niederschlag
                    \item Sonnenscheindauer
                    \item Sichtweite
                    \item Wind
                \end{itemize}
                Das waren die Unterteilungen der Messwerte auf Seite des DWD

            \subsubsection*{Zeiträume}
                Da unter jeder der eben aufgezählten Sektionen in dem Dateisystem des Servers eine Unterteilung in "recent" und "historical" stattfand und der gewünschte Zeitraum das Jahr 2022 war, galt es herauszufinden, wie die Aufteilung zustande kam. Leider gibt es keine klare Unterteilung und somit blieb nichts anders übrig, als in beiden Ordnern nachzuschauen.
            \subsubsection*{Dateien}
                In den zeitunterteilten Ordnern liegen genau eine Datei, die die Liste der Stationen mit ihren Eigenschaften aufzählt, welche den betrachteten Wert aufzeichnen sollen und hunderte komprimierte Ordner, benannt nach Stationscode, Daten der Erfassung (nur in "historical", nicht in "recent") und erfasste Eigenschaft. Teilweise reichten die erfassten Daten von 1970 bis 2001 oder von Mitte 2022 bis Anfang 2023, es war also kein System hinter den Aufzeichnungen zu Erkennen.
                Die ".zip" Ordner enthalten mehrere Dateien, teilweise ".html" und teilweise ".txt" Dateien und das sogar manchmal in doppelter Ausführung. Trotz alledem gibt es in jedem Ordner eine ausschlaggebende ".txt" Datei, die die vom Ordnernamen versprochenen Daten beinhalten.
        \\
        
        In den angesprochenen Dateien ist allerdings eine gewohnte Struktur wiederzufinden. Aufgebaut wie eine "csv" Datei nur mit Semikola getrennte Spalten. Die erste Spalte besteht aus dem Datum gepaart mit der Stundenzahl des Tages im 24h-Format, zu welchem Zeitpunkt die Daten der Reihe von den Messinstrumenten ausgelesen wurden. Die nächsten Spalten (1-3, je nach Datei) bestehen aus den gewünschten Attributen wie Sonnenscheindauer. Das Ende der Zeile macht immer ein "eor" als Zeilenende-Indikator.
        Manche Werte, unterschiedlich von Station zu Station und je nach Tageszeit sind gar nicht aufgelistet bzw. als fehlend (-999 in der Datei) gekennzeichnet. Für einige Stationen, wie zuvor schon thematisiert, stehen überhaupt keine Daten für bestimmte Messattribute in Form von Dateien in dem gewünschten Zeitraum zur Verfügung.
        Oft wurden auch nur in einem Teil des Jahres 2022 Daten erhoben, bzw. zur Veröffentlichung auf dem Server freigegeben, was zu "halben" Dateien führt.

        Trotz all diesen Hürden war es klar, dass das Sammeln dieser Daten nicht "per Hand" realisierbar ist, sondern durch Automatisierung geschehen soll. Ohne Vorkenntnisse in Python war das Vorhaben in der Tat anspruchsvoll, aber die überaus breit gefächerte Dokumentation der Sprache und den vielen Bibliotheken erleichterten das Programmieren sehr.
        Zu den Vorlesungen zu erscheinen, schien sich in diesem Zuge zu lohnen, als die Benutzung von RegEx (und somit den Grundlagen der Informatik) das Filtern der Dateien um einiges erleichterte. Einmal geschrieben, iterierte das Skript durch alle oben genannten Datenarten und sammelte und entpackte die Dateien auf der eigenen Festplatte. 
        Die Wetterdaten vollständig heruntergeladen blieb noch die Daten der Bundesnetzagentur.



\section*{Datenaufbereitung}
    Der nächste Schritt ist die Datenaufbereitung, bei der die eben gesammelten Daten in vorzugsweise ".csv" Dateien geschrieben/konvertiert werden.

    Zuerst wurden zwei eigene Dateien erstellt, die dann später noch raffiniert und letztendlich sogar partiell vereinigt wurden.

    \subsection*{Stromdaten}
        Die Daten der Bundesnetzagentur tabellarisch zu sortieren war keine Kunst, da es glücklicherweise passende Python Bibliotheken gibt, die ".json" Dateien in Listen einlesen können und erleichtern somit die Erstellung einer Datei im ".csv" Format.
        
        Ein Beschneiden der Daten war nun noch zu erledigen, da wie oben genannt die Epoch Codes nicht genau passten. Also wurden die überflüssigen Codes (die vor dem 01.01.2022 und die nach dem 12.31.2022) abgeschnitten. Dann wurden die Epoch Codes mit passender Bibliotheken in Daten übersetzt und eine "header"-Zeile wurde der Datei vorangesetzt. Durch die Epoch Codes konnte das Zeitumstellungssystem in Deutschland ignoriert werden, da die Unix-Zeit das nicht beachtet.
    \subsection*{Wetterdaten}
        Eine andere Handhabung gebührt hier den Wetterdaten. Das Komplexe an diesen Daten ist die Aufteilung. Gemeint damit, ist die hohe Anzahl an Stationen, die nicht immer die Gleichen/gleich viele Messwerte besitzen. Wie soll also eine Datenbank daraus kreiert werden. Das Vorgehen war hier essentiell, denn die gleichen Messwerte unterschiedlicher Stationen sollten nebeneinander liegen, aber es sind wie eben genannt nicht immer gleich viele für die einzelnen Werte.
        Folglich ist entschieden worden, dass alle Stationen in eine ".csv" Datei geschrieben werden und wenn es für eine Station gar keine Werte gibt, wird dies mit dem Wert -777 an der respektiven Stelle in der Datei vermerkt.

        Bevor das aber geschehen kann, wurden die ".txt" Dateien die für jede Station vorliegen zuerst in ".csv" Dateien geschrieben, um später eine einheitliche Verschmelzung garantieren zu können. So wurde für jede Station ein Ordner angelegt, in dem im besten Falle 6 Dateien gespeichert wurden. Je nach Attribut wurden aus den Textdateien unterschiedliche Spalten ausgelesen. Um den Zeitfaktor in Takt zu halten, bildete die erste Zeile jeder ".csv" immer Datum und Stundenwert ab.
        
        Um dann die große allumfassende tabellarisch strukturierte Datei aufzubauen wurde mit RegExs die Dateistruktur durchsucht und die Namen der Stationen wurden somit für die "header"-Zeile und für die Anzahl der Stationen erhalten. So konnten dann die Werte der kleinen spezifischen ".csv" Dateien in die große übernommen werden. All diese Operationen wurden natürlich auch per Python Skript verübt.

        \subsubsection*{Qualität}
            Nachdem das Übernehmen der Werte funktioniert hatte, reichte ein Blick, um festzustellen, dass die Datenlage nicht besonders qualitativ war.
            
            In, dem Gefühl nach, der Hälfte der Einträge war entweder eine -999 oder eine -777. Tatsächlich auch nicht zu verwunderlich, da von den über eintausend Stationen, mehr als 80\% nicht alle 6 Datentypen aufwiesen.
            
            Mit diesen Daten war nicht zu arbeiten, das stand fest. Die Frage war nur: "Wie kann die Datenlage verbessert werden?".
            Und genau dieses Problem wurde durch ein erweitertes Kultivieren der Daten adressiert.

    \subsection*{Verfeinerung}
        Speziell die Wetterdaten waren so nicht zu gebrauchen. Aber wie sollen die Daten beschränkt/verbessert werden, um eine repräsentative Menge zu erreichen, die auch noch das ganze Jahr 2022 abdeckt? Diese Frage war die zentrale Problemstellung der Datenaufbereitung für dieses Projekt. Und die Antwort darauf war nicht einfach zu finden. Erst einmal war es nicht möglich, die Datenlage ohne Begrenzung der Daten zu verbessern, da bereits alle vom DWD bereitgestellten und gewünschten Daten für diesen Zeitraum berücksichtigt wurden.
        Also war die andere und wohl einzig verbleibende Option die Begrenzung auf eine bestimmte Menge der Daten. Bei dieser Aufgabenstellung war der erste Gedanke, die Stationen für ein großes Dokument zu nutzen, die mindestens 5 von den 6 Wetterdatentypen aufweisen. Einfacher gesagt als getan, da durch einen Fehler in dem Python Programm, das die Textdateien in ".csv" Dateien übersetzt hat, auch leere Dateien in den Stationsordnern umherirrten. Auch bei erneutem Überarbeiten des Quellcodes war nicht klar, warum das geschieht. Also traten Dateien nur mit "header"-Zeile und ohne Daten auf, aber das ganz unregelmäßig und ohne zu erkennendes Muster. Trotzdem noch möglich, aber komplizierter als gedacht kam die Idee auf, einfach nur die Stationen zu benutzen, die alle Datentypen aufwiesen. Das Problem bestand dann immer noch, keine Frage, aber der Anteil der Fehlwerte sollte geringer sein. Fraglich dabei ist nur, ob die dann übergebliebene Anzahl an Stationen noch repräsentativ für das Wetter ist.

        Das konnte erst festgestellt werden, als ein, im Gegensatz zum Vorherigen, kleineres ".csv" Dokument erstellt wurde. Der Anblick war um einiges vielversprechender! 157 Stationen blieben nun noch übrig, die genau wie in der großen Datei geordnet, also den Wetterdatentypen nach und begonnen mit dem Datum und der Aufzeichnungsstunde in die Datei geschrieben wurden. Weitaus weniger, sowohl anteilsmäßig als natürlich auch absolut, Fehlwerte konnten mit bloßem Auge festgestellt werden. Die Menge an Daten war trotzdem noch sehr repräsentativ, da 157 Stationen á 10 Wetterattribute eine beachtliche Menge an Spalten ist.

        Trotzdem wurde festgestellt, dass es einen Bereich in dem Dokument gibt, der nur aus Fehlwerten besteht (-999). Das waren die Werte der Niederschlagsform. Auch in der Datenrecherche wurde bereits festgestellt, dass diese Spalte meist einfach nur Fehlwerte beinhaltet. Kurzerhand wurde entscheiden, dass dieser Indikator/Zahlencode keine besonders ausschlaggebende Bedeutung besitzt und er wurde in der nächste Verfeinerung der Dateien aus allen Stationen weggelassen.

        \subsubsection*{Strom als Additum für Weka}
            Anders als für den selbstgebauten Klassifikator, müssen für die Regression in Weka die Wetterdaten und dazugehörigen Stromdaten innerhalb einer Datei vorliegen, um sie sinnvoll zu verwerten. Da die erneuerbaren Energien Sonnenstrom und Windstrom essentiell waren, wurden 3 unterschiedliche Dateien kreiert. Allen diente die eben beschriebene Datei als Basis und als letzte Spalte wurde die Stromerzeugung in MWh angehängt. Einmal Solarstrom, einmal offshore Windkraft und einmal onshore Windkraft.

            Um das Testen unterschiedlicher Algorithmen in Weka schneller zu machen, wurden noch kompaktere Dateien erstellt, bei denen der Durchschnitt aller Wetterdaten als Wert für diesen Wetterdatentyp dient. Fehlwerte wurden nicht in den Durchschnitt mit einberechnet. Bei der Sonnenscheindauer waren oft zwischen 22Uhr und 3Uhr morgens, keine Aufzeichnungen bei allen 157 Stationen und dieser Wert wurde dann in der kompakteren Datei als realistische Abbildung auf 0 gesetzt.
            Auch an diese Dateien wurden dann analog zum vorherigen Prozess die Stromdaten als letzte Spalte angehängt.
        \\

        Um die Dateien dann noch schneller in Weka zu laden wurden alle ".csv" Dateien in ein ".arff" Format übersetzt, mit welchem sich Weka besser zurechtfindet.

\section*{Regressionanalyse mit Weka - Kay}
    Vorgehen beschreiben. Welche Modelle? Weka Version? Welche Files? Gab es Probleme? Auffällige Zeiten? Kompatiblität mit fehlwerten? Hat was gespackt uws. Sachen aus dem Weka Buch hilfreich?
\section*{Vergleich von Modellen - Kay}
    Was war denn eignetlich des beste und warum. Was war das Fehlermaß? Wie haben sie sich Zeitlich geschlagen? Weka gut/schlecht? Andere Ideen oder so. Wie war eigentlich Klassifikation - hätte das doch Sinn ergeben?
\section*{Fazit}
    Was kann aus der Arbeit geschlossen werden? Erfolg der Bearbeitung der Fragestellung? Aufwand/Zeit Verhältnis. Ergebnis der Modelle. Hätte man was besser machen können und wie?
    
\bibliography{WnPRegression}
\end{document}